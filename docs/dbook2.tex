\documentclass[]{book}
\usepackage{lmodern}
\usepackage{amssymb,amsmath}
\usepackage{ifxetex,ifluatex}
\usepackage{fixltx2e} % provides \textsubscript
\ifnum 0\ifxetex 1\fi\ifluatex 1\fi=0 % if pdftex
  \usepackage[T1]{fontenc}
  \usepackage[utf8]{inputenc}
\else % if luatex or xelatex
  \ifxetex
    \usepackage{mathspec}
  \else
    \usepackage{fontspec}
  \fi
  \defaultfontfeatures{Ligatures=TeX,Scale=MatchLowercase}
\fi
% use upquote if available, for straight quotes in verbatim environments
\IfFileExists{upquote.sty}{\usepackage{upquote}}{}
% use microtype if available
\IfFileExists{microtype.sty}{%
\usepackage{microtype}
\UseMicrotypeSet[protrusion]{basicmath} % disable protrusion for tt fonts
}{}
\usepackage[margin=1in]{geometry}
\usepackage{hyperref}
\hypersetup{unicode=true,
            pdftitle={D-score for international comparisons},
            pdfauthor={Stef van Buuren},
            pdfborder={0 0 0},
            breaklinks=true}
\urlstyle{same}  % don't use monospace font for urls
\usepackage{natbib}
\bibliographystyle{apalike}
\usepackage{longtable,booktabs}
\usepackage{graphicx,grffile}
\makeatletter
\def\maxwidth{\ifdim\Gin@nat@width>\linewidth\linewidth\else\Gin@nat@width\fi}
\def\maxheight{\ifdim\Gin@nat@height>\textheight\textheight\else\Gin@nat@height\fi}
\makeatother
% Scale images if necessary, so that they will not overflow the page
% margins by default, and it is still possible to overwrite the defaults
% using explicit options in \includegraphics[width, height, ...]{}
\setkeys{Gin}{width=\maxwidth,height=\maxheight,keepaspectratio}
\IfFileExists{parskip.sty}{%
\usepackage{parskip}
}{% else
\setlength{\parindent}{0pt}
\setlength{\parskip}{6pt plus 2pt minus 1pt}
}
\setlength{\emergencystretch}{3em}  % prevent overfull lines
\providecommand{\tightlist}{%
  \setlength{\itemsep}{0pt}\setlength{\parskip}{0pt}}
\setcounter{secnumdepth}{5}
% Redefines (sub)paragraphs to behave more like sections
\ifx\paragraph\undefined\else
\let\oldparagraph\paragraph
\renewcommand{\paragraph}[1]{\oldparagraph{#1}\mbox{}}
\fi
\ifx\subparagraph\undefined\else
\let\oldsubparagraph\subparagraph
\renewcommand{\subparagraph}[1]{\oldsubparagraph{#1}\mbox{}}
\fi

%%% Use protect on footnotes to avoid problems with footnotes in titles
\let\rmarkdownfootnote\footnote%
\def\footnote{\protect\rmarkdownfootnote}

%%% Change title format to be more compact
\usepackage{titling}

% Create subtitle command for use in maketitle
\newcommand{\subtitle}[1]{
  \posttitle{
    \begin{center}\large#1\end{center}
    }
}

\setlength{\droptitle}{-2em}
  \title{D-score for international comparisons}
  \pretitle{\vspace{\droptitle}\centering\huge}
  \posttitle{\par}
  \author{Stef van Buuren}
  \preauthor{\centering\large\emph}
  \postauthor{\par}
  \predate{\centering\large\emph}
  \postdate{\par}
  \date{2018-04-04}

\usepackage{booktabs}

\usepackage{amsthm}
\newtheorem{theorem}{Theorem}[chapter]
\newtheorem{lemma}{Lemma}[chapter]
\newtheorem{corollary}{Corollary}[chapter]
\newtheorem{proposition}{Proposition}[chapter]
\newtheorem{conjecture}{Conjecture}[chapter]
\theoremstyle{definition}
\newtheorem{definition}{Definition}[chapter]
\theoremstyle{definition}
\newtheorem{example}{Example}[chapter]
\theoremstyle{definition}
\newtheorem{exercise}{Exercise}[chapter]
\theoremstyle{remark}
\newtheorem*{remark}{Remark}
\newtheorem*{solution}{Solution}
\begin{document}
\maketitle

{
\setcounter{tocdepth}{1}
\tableofcontents
}
\chapter*{Preface}\label{preface}
\addcontentsline{toc}{chapter}{Preface}

This is an introductory booklet on the measurement of child development
by means of the D-score. The D-score is a one-number summary that
quantifies generic neurocognitive development for children with ages 0-4
years.

This is the \emph{second} in a series of three booklets. The series
consists of the following titles:

\begin{enumerate}
\def\labelenumi{\arabic{enumi}.}
\tightlist
\item
  \href{https://stefvanbuuren.github.io/dbook1/}{D-score for measuring
  development of children 0-4 years}
\item
  \href{https://stefvanbuuren.github.io/dbook2/}{D-score for
  international comparisons}
\item
  D-score for creating better instruments
\end{enumerate}

The development of this series is kindly supported by the Bill \&
Melinda Gates Foundation.

\chapter{Introduction}\label{ch:introduction2}

\section{SDG 4.2.1 indicator}\label{sdg-4.2.1-indicator}

\section{Quick scan of instruments}\label{quick-scan-of-instruments}

\section{Care-giver vs direct
assessment}\label{care-giver-vs-direct-assessment}

\section{Individual-programmatic-population
settings}\label{individual-programmatic-population-settings}

\chapter{Comparability}\label{ch:comparability}

\section{Challenge of comparability}\label{challenge-of-comparability}

\section{Bridging instruments by mapping
items}\label{bridging-instruments-by-mapping-items}

\section{Overview of promising item
mappings}\label{overview-of-promising-item-mappings}

\chapter{Equate groups}\label{ch:equategroups}

\section{Statistical framework for equate
groups}\label{statistical-framework-for-equate-groups}

\section{Common latent scale}\label{common-latent-scale}

\section{Differential item
functioning}\label{differential-item-functioning}

\section{Quantifying equate fit}\label{quantifying-equate-fit}

\chapter{Modeling equates}\label{ch:modelingequates}

\section{GCDG data: design and
description}\label{gcdg-data-design-and-description}

\section{Empirical and fitted item response
curves}\label{empirical-and-fitted-item-response-curves}

\section{Active and non-active equate
groups}\label{active-and-non-active-equate-groups}

\section{Splitting and combining equate
groups}\label{splitting-and-combining-equate-groups}

\section{modeling strategies}\label{modeling-strategies}

\chapter{Comparing ability}\label{ch:ability}

\section{D-score distribution by
study}\label{d-score-distribution-by-study}

\section{Impact of measurement error}\label{impact-of-measurement-error}

\chapter{Validity}\label{ch:validity2}

\section{Discriminatory validity}\label{discriminatory-validity}

\section{Concurrent validity}\label{concurrent-validity}

\section{Predictive validity}\label{predictive-validity}

\chapter{SDG 4.2.1 Indicator}\label{ch:SDGindicator}

\section{Application I: Estimating the SDG 4.2.1 indicator from existing
data}\label{application-i-estimating-the-sdg-4.2.1-indicator-from-existing-data}

\section{Data sets}\label{data-sets}

\section{Equate groups}\label{equate-groups}

\section{\texorpdfstring{Definition \emph{developmentally on
track}}{Definition developmentally on track}}\label{definition-developmentally-on-track}

\section{Country-level estimates}\label{country-level-estimates}

\section{Relation to other estimates}\label{relation-to-other-estimates}

\chapter{Who is on-track?}\label{ch:ontrack}

\section{Application II: What determines who is developmentally
on-track?}\label{application-ii-what-determines-who-is-developmentally-on-track}

\section{Types of explanatory
factors}\label{types-of-explanatory-factors}

\section{Relative importance}\label{relative-importance}

\section{Opportunities for
intervention}\label{opportunities-for-intervention}

\chapter{Discussion}\label{ch:discussion2}

\section{Potential of existing data for SDG
4.2.1}\label{potential-of-existing-data-for-sdg-4.2.1}

\section{Options for improving health
policy}\label{options-for-improving-health-policy}

\section{Suitability of D-score
metric}\label{suitability-of-d-score-metric}

\section{Suggestions for better
measurement}\label{suggestions-for-better-measurement}

\bibliography{book.bib}


\end{document}
